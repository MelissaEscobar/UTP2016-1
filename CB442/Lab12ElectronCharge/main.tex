\documentclass{article}
\usepackage[utf8]{inputenc}
\usepackage{graphicx}
\usepackage{amsmath}
\usepackage[margin=1in]{geometry}
\usepackage[spanish,es-tabla]{babel}
\usepackage{mathtools}


\title{Laboratorio 12. Medición de la carga del electrón}
\author{Carlos Alberto Dagua Conda, Hector Fabio Jimenez Saldarriaga, \\Juan Camilo
Castrillón,\thanks{carlosdaguaco@utp.edu.co, hfjimenez@utp.edu.co, jucacastrillon@utp.edu.co} }
\date{Abril 2016}

\begin{document}

\maketitle

\section{abstract}
In this paper we will determine the electron charge, we know that the electron is a fundamental particle with negative charge. A atom is composed by electrons, protons and neutrons, when the electrons exceed the energy level from atom this is called free electron. 

\section{Introducción}

Se dice que un objeto se encuentra cargado eléctricamente si sus átomos, mediante las interacciones electromagnéticas, adquieren una carga negativa o positiva. El flujo de la corriente a través de un conductor se da por medio del movimiento de los electrones libres, este flujo constante de electrones acarrea con el un efecto resistivo denominado \emph{efecto joule} y este se produce básicamente por la actividad electrónica en función de un tiempo $t$. 


\section{Análisis}
$\\$
\emph{Nota 1:} Debido a que el telescopio invierte la imagen de la gota, esta se
vio subir, cuando en realidad esta estaba bajando y viceversa.\newline
\newline
\emph{Nota 2:} Debido a que los relojes trabajan con una frecuencia de $50Hz$ y se trabajo en el laboratorio con $60Hz$ se utilizo un factor de corrección de $\frac{5}{6}$.


\begin{enumerate}
    \item Si la distancia que recorre la gota hacia arriba o abajo es de $3.0 mm$, calcule la velocidad de subida y bajada de cada gota (recuerde que se mueve con velocidad constante). Construya una tabla de datos de tiempos, velocidades y usando la ecuación $12.8$ adicione una columna con la carga de cada gota.\newline


$\bullet$ \emph{Solución: }\newline
Lo siguiente resultados corresponden a los datos de subida y bajada de las gotas de aceite observadas en el laboratorio, teniendo en cuenta que recorren una distancia de $3.00 mm$
\newline
Con lo anterior para una velocidad constante se tiene que,
\begin{equation}
    v=\frac{d}{t} \left[\frac{m}{s}\right]
\end{equation}
El cálculo de la carga de electrón está dada por la expresión,
\begin{equation}
    q=\frac{a}{V}\left( v_{g_{1}}+v_{g_{2}} \right)\sqrt{v_{g_{1}}+v_{g_{2}}}
\end{equation}

\begin{equation}
    a=\frac{9\pi}{4}\sqrt{\frac{\eta^{3}d^{2}}{\rho g}}
\end{equation}
$\\$
donde:
$\\$
\emph{d}: Distancia entre las placas del condensador= $3.0\times10^{-3}m$
\newline
\emph{$\eta$}: Viscosidad del aire= $1.855\times10^{-5}kg$
\newline
\emph{$\rho$}: Densidad del aceite= $851\frac{kg}{m^3}$\newline
\emph{g}: Aceleración de la gravedad= $9.8\frac{m}{s^{2}}$\newline
\emph{V}: $100V$\newline

(\emph{Véase tabla 2})



    \item Identifique la carga que tenga menor valor llámela $q_{1}$. Asuma que esta carga posee un electrón $(n_{1} = 1)$.
$\bullet$ \emph{Solución: }\newline
\begin{equation}
    q_{1}=7.3535E-20 (C),\ \ \ \ \ \ n=1
\end{equation}

    \item Divida las demás cargas entre el valor de $q_{1}$. Estos serán los valores de $n_{i}$ para cada dato de carga.\newline
    
$\bullet$ \emph{Solución: }\newline

(\emph{Véase tabla 3})

    \item Ubique los resultados en una tabla (en una columna el valor de la carga y en otra el resultado de la división entre $q_{1}$) y en orden ascendente de $q$. La carga $q_{1}$ va de primero en la tabla.\newline
    
$\bullet$ \emph{Solución: }\newline

(\emph{Véase tabla 3})



    \item Luego asuma que la carga de menor valor posee $2$ electrones $(n_{1} = 2)$. Divida las demás cargas entre $\frac{q_{1}}{n_{1}}$. Ubique los resultados en otra tabla repitiendo
    el proceso del ítem anterior.\newline
$\bullet$ \emph{Solución: }\newline

(\emph{Véase tabla 4})


    \item Repita el proceso anterior tomando $n_{1} = 3, 4, 5, ..., 10.$ Construya las respectivas
    tablas.\newline
$\bullet$ \emph{Solución: }\newline

(\emph{Véase tabla 5,6,7 y 8})


    \item En cada una de las tablas obtenidas calcule la diferencia entre cada $n_{i}$ y el entero mas cercano (en valor absoluto). Sume en cada tabla las diferencias de todos los $n_{i}$. Determine en cuál de las tablas esta sumatoria es la mínima.\newline
$\bullet$ \emph{Solución: }\newline

(\emph{Véase tabla 9})


    \item La tabla correspondiente a esta mínima diferencia contiene la información necesaria para calcular los valores de los enteros $n_{i}$ que le permiten despejar la carga del electrón utilizando la ecuación $12.10$. Tenga en cuenta que para calcular los $n_{i}$ debe aproximar los valores de su tabla al entero más cercano. Calcule entonces la carga del electrón para cada $n_{i}$.\newline
$\bullet$ \emph{Solución: }\newline

(\emph{Véase tabla 10})


    \item Como resultado final para la carga del electrón halle el promedio de los valores obtenidos en el punto anterior.\newline
$\bullet$ \emph{Solución: }\newline

\begin{equation}
    Valor \ promedio \ de \ e^{-}=1.2681E-19 \ C 
\end{equation}

    \item Calcule el error en su medida de la carga del electrón para cada gota, teniendo en cuenta que el error en la medida de las distancias es de $\pm 0.1 mm$, el error en la medida de los tiempos es de $\pm0.01 s$ y el error en 
    el valor de a es de un $10\%$ de $a$. Finalmente combine en cuadratura los errores obtenidos para encontrar la incertidumbre en la carga promedio electrónica.\newline
$\bullet$ \emph{Solución: }\newline

Para el cálculo de la incertidumbre en la carga del electrón, se procede a realizar la sumatoria del error porcentual para cada gota de aceite.

\begin{equation}
    \Delta e=2.61\%
\end{equation}

Para el calcular el error porcentual se procede de la siguiente forma:

\begin{equation}
    E_{\%}=\frac{|V_{teo}-V_{exp}|}{V_{teo}}\times 100
\end{equation}

\begin{equation}
    E_{\%}=\frac{|1.60217732\times 10^{-19}-1.26810310\times10^{-19}|}{1.60217732\times 10^{-19}}\times 100
\end{equation}

\begin{equation}
    E_{\%}=20.85\%
\end{equation}


    \item Compare el valor por usted hallado con el valor medido, y aceptado de: $1.60217732\times10^{-19}$\newline
$\bullet$ \emph{Solución: }\newline
\begin{equation}
    1.26810310\times10^{-19} < 1.60217732\times10^{-19} 
\end{equation}




\section{Conclusiones}

\begin{enumerate}
    \item Se determino la carga aproximada del electrón, con el método utilizado por Robert Millikan, teniendo como valor teórico el obtenido por Millikan cuando realizo su experimento.
    \item El error porcentual en la medida fue de un $20.85\%$ lo que permite un buen acercamiento al valor teórico.
    \item Se logro establecer que la carga del electrón se encuentra cuantizada.
\end{enumerate}

\section{Bibliografía}

$\\$
[1] \ \ Medición de la carga del electrón (2012), Universidad Tecnológica de Pereira. Tomado de:\\ http://media.utp.edu.co/facultad-ciencias-basicas  /archivos/contenidos
-departamento-de-fisica/\\
experimento12if.pdf


\newpage


\section{Apéndice}
\subsection{Tablas}

\begin{table}[h!]
    \centering
    \begin{tabular}{|c|c|c|}
    \hline
      Gota   & Subida (s) & Bajada (s)\\
      \hline
        1    &  5.83 & 10.69\\
        \hline
        2    &  15.03 & 10.06\\
        \hline
        3    &  5.26 & 6.28\\
        \hline
        4    &  3.84 & 3.87\\
        \hline
        5    &  4.49 & 4.52\\
        \hline
        6    &  6.73 & 5.69\\
        \hline
        7    &  12.89 & 12.13\\
        \hline
        8    &  17.79 & 20.08\\
        \hline
        9    &  5.13 & 7.68\\
        \hline
        10    &  8.35 & 7.50\\
        \hline
        11    &  8.48 & 10.33\\
        \hline
        12    &  6.12 & 5.10\\
        \hline
        13    &  7.75 & 8.43\\
        \hline
        14    &  5.55 & 3.72\\
        \hline
        15    &  11.93 & 7.19\\
        \hline
        16    &  16.47 & 11.63\\
        \hline
        17    &  10.87 & 10.78\\
        \hline
        18    &  12.03 & 10.45\\
        \hline
        19    &  20.53 & 14.38\\
        \hline
        20    &  24.72 & 20.45\\
        \hline
        21    &  11.42 & 7.58\\
        \hline
        22    &  20.92 & 16.55\\
        \hline
        23    &  6.50 & 2.92\\
        \hline
        24    &  16.01 & 12.20\\
        \hline
        25    &  15.51 & 13.74\\
        \hline
        
    \end{tabular}
    \caption{Datos tomados en el laboratorio.}
    \label{tab:my_label}
\end{table}


\begin{table}[h!]
    \centering
    \begin{tabular}{|c|c|c|c|c|c|}
    \hline
       Gota  & Subida (s) & Bajada (s) & V$_{subida}$ $(\frac{m}{s})$ & V$_{bajada}$ $(\frac{m}{s})$& Carga (C)\\ 
       \hline
       1  & 5.83 & 10.69 & 0.00017143 & 9.3531E-05 & 4.3473E-19 \\ 
       \hline
       2  & 10.06 & 15.03 & 0.00009940 & 6.6534E-05 & 1.7686E-19 \\
       \hline
       3  & 5.26 & 6.28 & 0.00019017 & 1.5915E-04 & 3.6170E-19 \\
       \hline
       4  & 3.84 & 3.87 & 0.00026030 & 2.5862E-04 & 1.2515E-19 \\       
       \hline
       5  & 4.49 & 4.52 & 0.00022291 & 2.2140E-04 & 1.0143E-19 \\       
       \hline
       6  & 5.69 & 6.73 & 0.00017575 & 1.4859E-04 & 3.1422E-19 \\       
       \hline
       7  & 11.30 & 12.89 & 0.00008850 & 7.7580E-05 & 1.0200E-19 \\       
       \hline
       8  & 19.46 & 20.08 & 0.00005139 & 4.9793E-05 & 2.3788E-20 \\       
       \hline
       9  & 5.13 & 7.68 & 0.00019481 & 1.3029E-04 & 4.8542E-19 \\       
       \hline
       10  & 7.50 & 8.35 & 0.00013333 & 1.1976E-04 & 1.7334E-19 \\   
       \hline
       11  & 8.48 & 10.33 & 0.00011788 & 9.6774E-05 & 1.8331E-19 \\        
       \hline
       12  & 5.10 & 6.12 & 0.00019608 & 1.6340E-04 & 3.8202E-19 \\        
       \hline
       13  & 7.75 & 8.43 & 0.00012903 & 1.1858E-04 & 1.4884E-19 \\        
       \hline
       14  & 3.72 & 5.55 & 0.00026882 & 1.8018E-04 & 7.8583E-19 \\        
       \hline
       15  & 7.19 & 11.93 & 0.00013908 & 8.3822E-05 & 3.0804E-19 \\        
       \hline
       16  & 11.63 & 16.47 & 0.00008598 & 6.0716E-05 & 1.3709E-19 \\        
       \hline
       17  & 10.78 & 10.87 & 0.00009276 & 9.1996E-05 & 3.0101E-20 \\ 
       \hline
       18  & 10.45 & 12.03 & 0.00009569 & 8.3126E-05 & 1.1785E-19 \\        
       \hline
       19  & 14.38 & 20.53 & 0.00006954 & 4.8709E-05 & 1.0033E-19 \\        
       \hline
       20  & 20.45 & 24.72 & 0.00004890 & 4.0453E-05 & 4.8276E-20 \\        
       \hline
       21  & 7.58 & 11.42 & 0.00013193 & 8.7566E-05 & 2.7177E-19 \\        
       \hline
       22  & 16.55 & 20.92 & 0.00006042 & 4.7801E-05 & 7.1477E-20 \\        
       \hline
       23  & 4.92 & 6.50 & 0.00020325 & 1.5385E-04 & 4.6661E-19 \\        
       \hline
       24  & 12.20 & 16.01 & 0.00008197 & 6.2461E-05 & 1.1858E-19 \\        
       \hline
       25  & 13.74 & 15.51 & 0.00007278 & 6.4475E-05 & 7.3535E-20 \\        
\hline
    \end{tabular}
    \caption{Cálculo de la velocidad de subida y de bajada y determinación de la carga asociada a cada gota.}
    \label{tab:my_label}
\end{table}



\begin{table}[h!]
    \centering
    \begin{tabular}{|c|c|c|}
    \hline
       N$_{i}$  & q$_{1}$ & $\frac{q_{i}}{q_{1}}$\\
       \hline
       n$_{1}$  & 7.3535E-20 & 5.9119E+00\\
       \hline
       n$_{2}$  & 7.3535E-20 & 2.4051E+00\\
       \hline       
       n$_{3}$  & 7.3535E-20 & 4.9188E+00\\
       \hline       
       n$_{4}$  & 7.3535E-20 & 1.7019E+00\\
       \hline       
       n$_{5}$  & 7.3535E-20 & 1.3794E+00\\
       \hline       
       n$_{6}$  & 7.3535E-20 & 4.2730E+00\\
       \hline       
       n$_{7}$  & 7.3535E-20 & 1.3871E+00\\
       \hline       
       n$_{8}$  & 7.3535E-20 & 3.2350E-01\\
       \hline       
       n$_{9}$  & 7.3535E-20 & 6.6012E+00\\
       \hline       
       n$_{10}$  & 7.3535E-20 & 2.3572E+00\\
       \hline       
       n$_{11}$  & 7.3535E-20 & 2.4929E+00\\
       \hline       
       n$_{12}$  & 7.3535E-20 & 5.1951E+00\\
       \hline       
       n$_{13}$  & 7.3535E-20 & 2.0240E+00\\
       \hline       
       n$_{14}$  & 7.3535E-20 & 1.0687E+01\\
       \hline
       n$_{15}$  & 7.3535E-20 & 4.1890E+00\\
       \hline       
       n$_{16}$  & 7.3535E-20 & 1.8643E+00\\
       \hline       
       n$_{17}$  & 7.3535E-20 & 4.0935E-01\\
       \hline       
       n$_{18}$  & 7.3535E-20 & 1.6026E+00\\
       \hline       
       n$_{19}$  & 7.3535E-20 & 1.3644E+00\\
       \hline       
       n$_{20}$  & 7.3535E-20 & 6.5650E-01\\
       \hline       
       n$_{21}$  & 7.3535E-20 & 3.6957E+00\\
       \hline       
       n$_{22}$  & 7.3535E-20 & 9.7201E-01\\
       \hline       
       n$_{23}$  & 7.3535E-20 & 6.3454E+00\\
       \hline       
       n$_{24}$  & 7.3535E-20 & 1.6126E+00\\
       \hline       
       n$_{25}$  & 7.3535E-20 & 1.0000E+00\\
       \hline

    \end{tabular}
    \caption{$n_{1}=1$, carga mínima cuantizada en cada gota.}
    \label{tab:my_label}
\end{table}


\begin{table}[h!]
    \centering
    \begin{tabular}{|c|c|c|}
    \hline
       N$_{i}$  & $\frac{q_{1}}{2}$ & $\frac{q_{i}}{q_{1}}$\\
       \hline
       n$_{1}$  & 3.6768E-20 & 1.1824E+01\\
       \hline
       n$_{2}$  & 3.6768E-20 & 4.8102E+00\\
       \hline       
       n$_{3}$  & 3.6768E-20 & 9.8376E+00\\
       \hline       
       n$_{4}$  & 3.6768E-20 & 3.4038E+00\\
       \hline       
       n$_{5}$  & 3.6768E-20 & 2.7587E+00\\
       \hline       
       n$_{6}$  & 3.6768E-20 & 8.5460E+00\\
       \hline       
       n$_{7}$  & 3.6768E-20 & 2.7743E+00\\
       \hline       
       n$_{8}$  & 3.6768E-20 & 6.4699E-01\\
       \hline       
       n$_{9}$  & 3.6768E-20 & 1.3202E+01\\
       \hline       
       n$_{10}$  & 3.6768E-20 & 4.7145E+00\\
       \hline       
       n$_{11}$  & 3.6768E-20 & 4.9858E+00\\
       \hline       
       n$_{12}$  & 3.6768E-20 & 1.0390E+01\\
       \hline       
       n$_{13}$  & 3.6768E-20 & 4.0481E+00\\
       \hline       
       n$_{14}$  & 3.6768E-20 & 2.1373E+01\\
       \hline
       n$_{15}$  & 3.6768E-20 & 8.3780E+00\\
       \hline       
       n$_{16}$  & 3.6768E-20 & 3.7285E+00\\
       \hline       
       n$_{17}$  & 3.6768E-20 & 8.1869E-01\\
       \hline       
       n$_{18}$  & 3.6768E-20 & 3.2053E+00\\
       \hline       
       n$_{19}$  & 3.6768E-20 & 2.7289E+00\\
       \hline       
       n$_{20}$  & 3.6768E-20 & 1.3130E+00\\
       \hline       
       n$_{21}$  & 3.6768E-20 & 7.3915E+00\\
       \hline       
       n$_{22}$  & 3.6768E-20 & 1.9440E+00\\
       \hline       
       n$_{23}$  & 3.6768E-20 & 1.2691E+01\\
       \hline       
       n$_{24}$  & 3.6768E-20 & 3.2252E+00\\
       \hline       
       n$_{25}$  & 3.6768E-20 & 2.0000E+00\\
       \hline

    \end{tabular}
    \caption{$n_{1}=2$, carga mínima cuantizada en cada gota.}
    \label{tab:my_label}
\end{table}



\begin{table}[h!]
    \centering
    \begin{tabular}{|c|c|c|c|c|}
    \hline
       N$_{i}$  & $\frac{q_{1}}{3}$ & $\frac{q_{i}}{q_{1}}$ & $\frac{q_{1}}{4}$ & $\frac{q_{i}}{q_{1}}$\\
       \hline
       n$_{1}$  & 2.4512E-20 & 1.7736E+01 & 1.8384E-20 & 2.3648E+01\\
       \hline
       n$_{2}$  & 2.4512E-20 & 7.2152E+00 & 1.8384E-20 & 9.6203E+00\\
       \hline       
       n$_{3}$  & 2.4512E-20 & 1.4756E+01 & 1.8384E-20 & 1.9675E+01\\
       \hline       
       n$_{4}$  & 2.4512E-20 & 5.1057E+00 & 1.8384E-20 & 6.8076E+00\\
       \hline       
       n$_{5}$  & 2.4512E-20 & 4.1381E+00 & 1.8384E-20 & 5.5174E+00\\
       \hline       
       n$_{6}$  & 2.4512E-20 & 1.2819E+01 & 1.8384E-20 & 1.7092E+01\\
       \hline       
       n$_{7}$  & 2.4512E-20 & 4.1614E+00 & 1.8384E-20 & 5.5486E+00\\
       \hline       
       n$_{8}$  & 2.4512E-20 & 9.7049E-01 & 1.8384E-20 & 1.2940E+00\\
       \hline       
       n$_{9}$  & 2.4512E-20 & 1.9803E+01 & 1.8384E-20 & 2.6405E+01\\
       \hline       
       n$_{10}$  & 2.4512E-20 & 7.0717E+00 & 1.8384E-20 & 9.4289E+00\\
       \hline       
       n$_{11}$  & 2.4512E-20 & 7.4787E+00 & 1.8384E-20 & 9.9716E+00\\
       \hline       
       n$_{12}$  & 2.4512E-20 & 1.5585E+01 & 1.8384E-20 & 2.0780E+01\\
       \hline       
       n$_{13}$  & 2.4512E-20 & 6.0721E+00 & 1.8384E-20 & 8.0961E+00\\
       \hline       
       n$_{14}$  & 2.4512E-20 & 3.2060E+01 & 1.8384E-20 & 4.2746E+01\\
       \hline
       n$_{15}$  & 2.4512E-20 & 1.2567E+01 & 1.8384E-20 & 1.6756E+01\\
       \hline       
       n$_{16}$  & 2.4512E-20 & 5.5928E+00 & 1.8384E-20 & 7.4570E+00\\
       \hline       
       n$_{17}$  & 2.4512E-20 & 1.2280E+00 & 1.8384E-20 & 1.6374E+00\\
       \hline       
       n$_{18}$  & 2.4512E-20 & 4.8079E+00 & 1.8384E-20 & 6.4106E+00\\
       \hline       
       n$_{19}$  & 2.4512E-20 & 4.0933E+00 & 1.8384E-20 & 5.4577E+00\\
       \hline       
       n$_{20}$  & 2.4512E-20 & 1.9695E+00 & 1.8384E-20 & 2.6260E+00\\
       \hline       
       n$_{21}$  & 2.4512E-20 & 1.1087E+01 & 1.8384E-20 & 1.4783E+01\\
       \hline       
       n$_{22}$  & 2.4512E-20 & 2.9160E+00 & 1.8384E-20 & 3.8880E+00\\
       \hline       
       n$_{23}$  & 2.4512E-20 & 1.9036E+01 & 1.8384E-20 & 2.5382E+01\\
       \hline       
       n$_{24}$  & 2.4512E-20 & 4.8378E+00 & 1.8384E-20 & 6.4504E+00\\
       \hline       
       n$_{25}$  & 2.4512E-20 & 3.0000E+00 & 1.8384E-20 & 4.0000E+00\\
       \hline

    \end{tabular}
    \caption{$n_{1}=3,4$, carga mínima cuantizada en cada gota.}
    \label{tab:my_label}
\end{table}



\begin{table}[h!]
    \centering
    \begin{tabular}{|c|c|c|c|c|}
    \hline
       N$_{i}$  & $\frac{q_{1}}{5}$ & $\frac{q_{i}}{q_{1}}$ & $\frac{q_{1}}{6}$ & $\frac{q_{i}}{q_{1}}$\\
       \hline
       n$_{1}$  & 1.4707E-20 & 2.9560E+01 & 1.2256E-20 & 3.5471E+01\\
       \hline
       n$_{2}$  & 1.4707E-20 & 1.2025E+01 & 1.2256E-20 & 1.4430E+01\\
       \hline       
       n$_{3}$  & 1.4707E-20 & 2.4594E+01 & 1.2256E-20 & 2.9513E+01\\
       \hline       
       n$_{4}$  & 1.4707E-20 & 8.5094E+00 & 1.2256E-20 & 1.0211E+01\\
       \hline       
       n$_{5}$  & 1.4707E-20 & 6.8968E+00 & 1.2256E-20 & 8.2761E+00\\
       \hline       
       n$_{6}$  & 1.4707E-20 & 2.1365E+01 & 1.2256E-20 & 2.5638E+01\\
       \hline       
       n$_{7}$  & 1.4707E-20 & 6.9357E+00 & 1.2256E-20 & 8.3229E+00\\
       \hline       
       n$_{8}$  & 1.4707E-20 & 1.6175E+00 & 1.2256E-20 & 1.9410E+00\\
       \hline       
       n$_{9}$  & 1.4707E-20 & 3.3006E+01 & 1.2256E-20 & 3.9607E+01\\
       \hline       
       n$_{10}$  & 1.4707E-20 & 1.1786E+01 & 1.2256E-20 & 1.4143E+01\\
       \hline       
       n$_{11}$  & 1.4707E-20 & 1.2464E+01 & 1.2256E-20 & 1.4957E+01\\
       \hline       
       n$_{12}$  & 1.4707E-20 & 2.5976E+01 & 1.2256E-20 & 3.1171E+01\\
       \hline       
       n$_{13}$  & 1.4707E-20 & 1.0120E+01 & 1.2256E-20 & 1.2144E+01\\
       \hline       
       n$_{14}$  & 1.4707E-20 & 5.3433E+01 & 1.2256E-20 & 6.4119E+01\\
       \hline
       n$_{15}$  & 1.4707E-20 & 2.0945E+01 & 1.2256E-20 & 2.5134E+01\\
       \hline       
       n$_{16}$  & 1.4707E-20 & 9.3213E+00 & 1.2256E-20 & 1.1186E+01\\
       \hline       
       n$_{17}$  & 1.4707E-20 & 2.0467E+00 & 1.2256E-20 & 2.4561E+00\\
       \hline       
       n$_{18}$  & 1.4707E-20 & 8.0132E+00 & 1.2256E-20 & 9.6159E+00\\
       \hline       
       n$_{19}$  & 1.4707E-20 & 6.8221E+00 & 1.2256E-20 & 8.1866E+00\\
       \hline       
       n$_{20}$  & 1.4707E-20 & 3.2825E+00 & 1.2256E-20 & 3.9390E+00\\
       \hline       
       n$_{21}$  & 1.4707E-20 & 1.8479E+01 & 1.2256E-20 & 2.2174E+01\\
       \hline       
       n$_{22}$  & 1.4707E-20 & 4.8600E+00 & 1.2256E-20 & 5.8321E+00\\
       \hline       
       n$_{23}$  & 1.4707E-20 & 3.1727E+01 & 1.2256E-20 & 3.8073E+01\\
       \hline       
       n$_{24}$  & 1.4707E-20 & 8.0630E+00 & 1.2256E-20 & 9.6755E+00\\
       \hline       
       n$_{25}$  & 1.4707E-20 & 5.0000E+00 & 1.2256E-20 & 6.0000E+00\\
       \hline

    \end{tabular}
    \caption{$n_{1}=5,6$, carga mínima cuantizada en cada gota.}
    \label{tab:my_label}
\end{table}


\begin{table}[h!]
    \centering
    \begin{tabular}{|c|c|c|c|c|}
    \hline
       N$_{i}$  & $\frac{q_{1}}{7}$ & $\frac{q_{i}}{q_{1}}$ & $\frac{q_{1}}{8}$ & $\frac{q_{i}}{q_{1}}$\\
       \hline
       n$_{1}$  & 1.0505E-20 & 4.1383E+01 & 7.3535E-21 & 5.9119E+01\\
       \hline
       n$_{2}$  & 1.0505E-20 & 1.6836E+01 & 7.3535E-21 & 2.4051E+01\\
       \hline       
       n$_{3}$  & 1.0505E-20 & 3.4432E+01 & 7.3535E-21 & 4.9188E+01\\
       \hline       
       n$_{4}$  & 1.0505E-20 & 1.1913E+01 & 7.3535E-21 & 1.7019E+01\\
       \hline       
       n$_{5}$  & 1.0505E-20 & 9.6555E+00 & 7.3535E-21 & 1.3794E+01\\
       \hline       
       n$_{6}$  & 1.0505E-20 & 2.9911E+01 & 7.3535E-21 & 4.2730E+01\\
       \hline       
       n$_{7}$  & 1.0505E-20 & 9.7100E+00 & 7.3535E-21 & 1.3871E+01\\
       \hline       
       n$_{8}$  & 1.0505E-20 & 2.2645E+00 & 7.3535E-21 & 3.2350E+00\\
       \hline       
       n$_{9}$  & 1.0505E-20 & 4.6208E+01 & 7.3535E-21 & 6.6012E+01\\
       \hline       
       n$_{10}$  & 1.0505E-20 & 1.6501E+01 & 7.3535E-21 & 2.3572E+01\\
       \hline       
       n$_{11}$  & 1.0505E-20 & 1.7450E+01 & 7.3535E-21 & 2.4929E+01\\
       \hline       
       n$_{12}$  & 1.0505E-20 & 3.6366E+01 & 7.3535E-21 & 5.1951E+01\\
       \hline       
       n$_{13}$  & 1.0505E-20 & 1.4168E+01 & 7.3535E-21 & 2.0240E+01\\
       \hline       
       n$_{14}$  & 1.0505E-20 & 7.4806E+01 & 7.3535E-21 & 1.0687E+02\\
       \hline
       n$_{15}$  & 1.0505E-20 & 2.9323E+01 & 7.3535E-21 & 4.1890E+01\\
       \hline       
       n$_{16}$  & 1.0505E-20 & 1.3050E+01 & 7.3535E-21 & 1.8643E+01\\
       \hline       
       n$_{17}$  & 1.0505E-20 & 2.8654E+00 & 7.3535E-21 & 4.0935E+00\\
       \hline       
       n$_{18}$  & 1.0505E-20 & 1.1219E+01 & 7.3535E-21 & 1.6026E+01\\
       \hline       
       n$_{19}$  & 1.0505E-20 & 9.5510E+00 & 7.3535E-21 & 1.3644E+01\\
       \hline       
       n$_{20}$  & 1.0505E-20 & 4.5955E+00 & 7.3535E-21 & 6.5650E+00\\
       \hline       
       n$_{21}$  & 1.0505E-20 & 2.5870E+01 & 7.3535E-21 & 3.6957E+01\\
       \hline       
       n$_{22}$  & 1.0505E-20 & 6.8041E+00 & 7.3535E-21 & 9.7201E+00\\
       \hline       
       n$_{23}$  & 1.0505E-20 & 4.4418E+01 & 7.3535E-21 & 6.3454E+01\\
       \hline       
       n$_{24}$  & 1.0505E-20 & 1.1288E+01 & 7.3535E-21 & 1.6126E+01\\
       \hline       
       n$_{25}$  & 1.0505E-20 & 7.0000E+00 & 7.3535E-21 & 1.0000E+01\\
       \hline

    \end{tabular}
    \caption{$n_{1}=7,8$, carga mínima cuantizada en cada gota.}
    \label{tab:my_label}
\end{table}



\begin{table}[h!]
    \centering
    \begin{tabular}{|c|c|c|c|c|}
    \hline
       N$_{i}$  & $\frac{q_{1}}{9}$ & $\frac{q_{i}}{q_{1}}$ & $\frac{q_{1}}{10}$ & $\frac{q_{i}}{q_{1}}$\\
       \hline
       n$_{1}$  & 8.1706E-21 & 5.3207E+01 & 9.1919E-21 & 4.7295E+01\\
       \hline
       n$_{2}$  & 8.1706E-21 & 2.1646E+01 & 9.1919E-21 & 1.9241E+01\\
       \hline       
       n$_{3}$  & 8.1706E-21 & 4.4269E+01 & 9.1919E-21 & 3.9350E+01\\
       \hline       
       n$_{4}$  & 8.1706E-21 & 1.5317E+01 & 9.1919E-21 & 1.3615E+01\\
       \hline       
       n$_{5}$  & 8.1706E-21 & 1.2414E+01 & 9.1919E-21 & 1.1035E+01\\
       \hline       
       n$_{6}$  & 8.1706E-21 & 3.8457E+01 & 9.1919E-21 & 3.4184E+01\\
       \hline       
       n$_{7}$  & 8.1706E-21 & 1.2484E+01 & 9.1919E-21 & 1.1097E+01\\
       \hline       
       n$_{8}$  & 8.1706E-21 & 2.9115E+00 & 9.1919E-21 & 2.5880E+00\\
       \hline       
       n$_{9}$  & 8.1706E-21 & 5.9410E+01 & 9.1919E-21 & 5.2809E+01\\
       \hline       
       n$_{10}$  & 8.1706E-21 & 2.1215E+01 & 9.1919E-21 & 1.8858E+01\\
       \hline       
       n$_{11}$  & 8.1706E-21 & 2.2436E+01 & 9.1919E-21 & 1.9943E+01\\
       \hline       
       n$_{12}$  & 8.1706E-21 & 4.6756E+01 & 9.1919E-21 & 4.1561E+01\\
       \hline       
       n$_{13}$  & 8.1706E-21 & 1.8216E+01 & 9.1919E-21 & 1.6192E+01\\
       \hline       
       n$_{14}$  & 8.1706E-21 & 9.6179E+01 & 9.1919E-21 & 8.5492E+01\\
       \hline
       n$_{15}$  & 8.1706E-21 & 3.7701E+01 & 9.1919E-21 & 3.3512E+01\\
       \hline       
       n$_{16}$  & 8.1706E-21 & 1.6778E+01 & 9.1919E-21 & 1.4914E+01\\
       \hline       
       n$_{17}$  & 8.1706E-21 & 3.6841E+00 & 9.1919E-21 & 3.2748E+00\\
       \hline       
       n$_{18}$  & 8.1706E-21 & 1.4424E+01 & 9.1919E-21 & 1.2821E+01\\
       \hline       
       n$_{19}$  & 8.1706E-21 & 1.2280E+01 & 9.1919E-21 & 1.0915E+01\\
       \hline       
       n$_{20}$  & 8.1706E-21 & 5.9085E+00 & 9.1919E-21 & 5.2520E+00\\
       \hline       
       n$_{21}$  & 8.1706E-21 & 3.3262E+01 & 9.1919E-21 & 2.9566E+01\\
       \hline       
       n$_{22}$  & 8.1706E-21 & 8.7481E+00 & 9.1919E-21 & 7.7761E+00\\
       \hline       
       n$_{23}$  & 8.1706E-21 & 5.7109E+01 & 9.1919E-21 & 5.0764E+01\\
       \hline       
       n$_{24}$  & 8.1706E-21 & 1.4513E+01 & 9.1919E-21 & 1.2901E+01\\
       \hline       
       n$_{25}$  & 8.1706E-21 & 9.0000E+00 & 9.1919E-21 & 8.0000E+00\\
       \hline

    \end{tabular}
    \caption{$n_{1}=9,10$, carga mínima cuantizada en cada gota.}
    \label{tab:my_label}
\end{table}


\begin{table}[h!]
    \centering
    
    \begin{tabular}{|c|c|}
\hline    
       n$_{i}$ & Valor aproximado\\
       \hline
       2.7134E+00 & 3\\
       \hline
       1.1039E+00 & 1\\
       \hline
       2.2576E+00 & 2\\
       \hline
       7.8112E-01 & 1\\
       \hline
       6.3308E-01 & 1\\
       \hline
       1.9612E+00 & 2\\
       \hline
       6.3666E-01 & 1\\
       \hline
       1.4847E-01 & 1\\
       \hline
       3.0297E+00 & 3\\
       \hline
       1.0819E+00 & 1\\
       \hline
       1.1442E+00 & 1\\
       \hline
       2.3844E+00 & 2\\
       \hline
       9.2897E-01 & 1\\
       \hline
       4.9048E+00 & 5\\
       \hline       
       1.9226E+00 & 2\\
       \hline       
       8.5563E-01 & 1\\
       \hline       
       1.8788E-01 & 1\\
       \hline       
       7.3557E-01 & 1\\
       \hline       
       6.2623E-01 & 1\\
       \hline       
       3.0131E-01 & 1\\
       \hline       
       1.6962E+00 & 2\\
       \hline       
       4.4612E-01 & 1\\
       \hline       
       2.9124E+00 & 3\\
       \hline       
       7.4013E-01 & 1\\
       \hline 
       4.5897E-01 & 1\\
       \hline
  
    \end{tabular}
    \caption{Diferencia entre los $n_{i}$ cuantizados y valor aproximado de $n_{i}$ para cada gota.}
    \label{tab:my_label}
\end{table}


\begin{table}[h!]
    \centering
    \begin{tabular}{|c|c|c|}
    \hline
      Carga $e^{-}$   & Valor aproximado &  $e=q_{1}/valor$\\
      \hline
       4.3473E-19  &  3 & 1.4491E-19 \\
       \hline
       1.7686E-19  &  1 & 1.7686E-19 \\
       \hline       
       3.6170E-19  &  2 & 1.8085E-19 \\
       \hline       
       1.2515E-19  &  1 & 1.2515E-19 \\
       \hline       
       1.0143E-19  &  1 & 1.0143E-19 \\
       \hline       
       3.1422E-19 &  2 & 1.5711E-19 \\
       \hline       
       1.0200E-19  &  1 & 1.0200E-19 \\
       \hline       
       2.3788E-20  &  1 & 2.3788E-20 \\
       \hline       
       4.8542E-19  &  3 & 1.6181E-19 \\
       \hline       
       1.7334E-19  &  1 & 1.7334E-19 \\
       \hline       
       1.8331E-19  &  1 & 1.8331E-19 \\
       \hline
       3.8202E-19  &  2 & 1.9101E-19 \\
       \hline       
       1.4884E-19  &  1 & 1.4884E-19 \\
       \hline       
       7.8583E-19  &  5 & 1.5717E-19\\
       \hline       
       3.0804E-19  &  2 & 1.5402E-19 \\
       \hline       
       1.3709E-19  &  1 & 1.3709E-19 \\
       \hline       
       3.0101E-20  &  1 & 3.0101E-20 \\
       \hline       
       1.1785E-19  &  1 & 1.1785E-19 \\
       \hline       
       1.0033E-19 &  1 & 1.0033E-19 \\
       \hline     
       4.8276E-20 &  1 & 4.8276E-20\\
       \hline        
       2.7177E-19 &  2 & 1.3588E-19 \\
       \hline        
       7.1477E-20 &  1 & 7.1477E-20 \\
       \hline        
       4.6661E-19 &  4 & 1.5554E-19 \\
       \hline   
       1.1858E-19 &  1 & 1.1858E-19 \\
       \hline         
       7.3535E-20 &  1 & 7.3535E-20 \\
       \hline         

    \end{tabular}
    \caption{Cálculo de la carga del electrón y el valor $n_{i}$ asociado.}
    \label{tab:my_label}
\end{table}


\end{enumerate}
\end{document}
