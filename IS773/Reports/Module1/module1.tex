%%%%%%%%%%%%%%%%%%%%%%%%%%%%%%%%%%%%%%%%%
% Original author:
% Linux and Unix Users Group at Virginia Tech Wiki
% (https://vtluug.org/wiki/Example_LaTeX_chem_lab_report)
% Modified by: Hector F. Jimenez S.
% License:
% CC BY-NC-SA 3.0 (http://creativecommons.org/licenses/by-nc-sa/3.0/)
%%%%%%%%%%%%%%%%%%%%%%%%%%%%%%%%%%%%%%%%%
%----------------------------------------
%	PACKAGES AND DOCUMENT CONFIGURATIONS
%---------------------------------------
\documentclass{article}
\usepackage[version=3]{mhchem}          % Package for chemical equation typesetting
\usepackage{siunitx}                    % Provides the \SI{}{} and \si{} command for typesetting SI units
\usepackage{graphicx}                   % Required for the inclusion of images
\usepackage{natbib}                     % Required to change bibliography style to APA
\usepackage{amsmath}                    % Required for some math elements
\usepackage[spanish]{babel}
\usepackage[utf8]{inputenc}             %

\setlength\parindent{0pt}               % Removes all indentation from paragraphs
\renewcommand{\labelenumi}{\alph{enumi}.} % Make numbering in the enumerate environment by letter rather than number (e.g. section 6)
\usepackage{times}
%------------------------
%	DOCUMENT INFORMATION
%------------------------
\title{Desarrollo de un Controlador de Tráfico\\ Usando FPGA's \\ Laboratorio de Electrónica Digital\\Modulo: 1} % Title
\author{Héctor F. \textsc{Jiménez Saldarriaga}\\ hfjimenez@utp.edu.co} % Author name
\date{\today}                           % Date for the report
\begin{document}
\maketitle                              % Insert the title, author and date
\begin{center}
\begin{tabular}{l r}
Fecha de Entrega: & Febrero 24, 2016 \\
Profesor: & Ing.Msc(c) Ramiro Andres Barrios Valencia
\end{tabular}
\end{center}
% \begin{abstract}
% Abstract text
% \end{abstract}
%------------------
%	SECTION 1
%------------------
\section{Objectivos}
\begin{enumerate}
  \item Fortalecer y poner en práctica la teoria de circuitos combinacionales para la solución al problema.
  \item Fortalecer el desarrllo de sistemas digitales, utilizando el lenguaje \emph{VHDL} y su entorno de desarrollo.
  \item Identificar la arquitectura y el hardware integrado de la placa de desarrollo NEXYS2.
  \item Implementar en lenguaje VHDL un modúlo que permita la detección correcta de los estados lógicos de los botones presentes en la fpga modelo \emph{NEXYS 2}\footnote{Datasheet Oficial de la placa: http://reference.digilentinc.com/_media/nexys:nexys2:nexys2_rm.pdf}
\end{enumerate}

% If you have more than one objective, uncomment the below:
%\begin{description}
%\item[First Objective] \hfill \\
%Objective 1 text
%\item[Second Objective] \hfill \\
%Objective 2 text
%\end{description}

\section{Marco Teorico}

\subsubsection{Arquitectura del Modelo\emph{NEXYS 2}}
El modelo \emph{NEXYS 2} esta governado 
%------------------
%	SECTION 2
%------------------
\section{Simulación}
Para poner a prueba el diseño de este modo se utilizo un simulador, el cual arrojara en detalle los resultados del modulo funcional.
%------------------
%	SECTION 3
%------------------
\section{Prácticas Experimentales}
Poner a qui la practica realizada en laboratorio.
%------------------
%	SECTION 3
%------------------
\section{Explicación Código VHDL}
Because of this reaction, the required ratio is the atomic weight of magnesium: \SI{16.00}{\gram} of oxygen as experimental mass of Mg: experimental mass of oxygen or $\frac{x}{1.31}=\frac{16}{0.87}$ from which, $M_{\ce{Mg}} = 16.00 \times \frac{1.31}{0.87} = 24.1 = \SI{24}{\gram\per\mole}$ (to two significant figures).

\section{Problemas Hallados en el Desarrollo}
The atomic weight of magnesium is concluded to be \SI{24}{\gram\per\mol}, as determined by the stoichiometry of its chemical combination with oxygen. This result is in agreement with the accepted value.

\begin{figure}[h]
\begin{center}
\includegraphics[width=0.65\textwidth]{placeholder} % Include the image placeholder.png
\caption{Figure caption.}
\end{center}
\end{figure}
%------------------
%	SECTION 5
%------------------
\section{Resultados y Conclusiones}
\begin{enumerate}
\begin{item}
The \emph{atomic weight of an element} is the relative weight of one of its atoms compared to C-12 with a weight of 12.0000000$\ldots$, hydrogen with a weight of 1.008, to oxygen with a weight of 16.00. Atomic weight is also the average weight of all the atoms of that element as they occur in nature.
\end{item}
\begin{item}
The \emph{units of atomic weight} are two-fold, with an identical numerical value. They are g/mole of atoms (or just g/mol) or amu/atom.
\end{item}
\begin{item}
\emph{Percentage discrepancy} between an accepted (literature) value and an experimental value is
\begin{equation*}
\frac{\mathrm{experimental\;result} - \mathrm{accepted\;result}}{\mathrm{accepted\;result}}
\end{equation*}
\end{item}
\end{enumerate}
%------------------
%	BIBLIOGRAPHY
%------------------
\bibliographystyle{apalike}
\bibliography{recursos}
%------------------
\end{document}